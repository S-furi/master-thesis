%!TEX root = ../thesis-main.tex
%\centerimage{./figures/Basic-Usecase.png}{Diagramma dei casi d'uso del sistema di \ac{API} per Alchemist}{basic-usecase}{0.6}
\newcommand{\centerimage}[4] {
	\begin{figure}[htb]
		\begin{center}
			\includegraphics[width=#4\linewidth]{#1}
			\caption{#2}
			\label{fig:#3}
		\end{center}
	\end{figure}
}

%\centerimagesource{figures/alchemist-model.png}{Meta-modello di Alchemist}{alchemist-metamodel}{1}{https://alchemistsimulator.github.io/explanation/metamodel/}
\newcommand{\centerimagesource}[5] {
	\begin{figure}[htb]
		\begin{center}
            \def\stackalignment{r}
            \stackunder{\includegraphics[width=#4\linewidth]{#1}}%
            {\scriptsize%
            Source: \url{#5}}
			\caption{#2}
			\label{fig:#3}
		\end{center}
	\end{figure}
}

\setmintedinline{breaklines}

\newcommand{\kotlin}[1]{
	\mintinline[breakanywhere]{kotlin}{#1}
}

\newcommand{\java}[1]{
	\mintinline[breakanywhere]{java}{#1}
}

\newcommand*{\code}[4]{
	\begin{listing}[htb]
		\inputminted[
		frame=lines,
		framesep=2mm,
		baselinestretch=1.2,
		bgcolor=LightGray,
		linenos
		]{#1}{#2}
		\caption{#3}
		\label{lst:#4}
	\end{listing}
}

% \includediagram[width]{filename}{caption}{label}
\newcommand{\includediagram}[4][\linewidth]{%
    \begin{figure}[htbp]
        \centering
        \resizebox{#1}{!}{%
            \input{#2}%
        }%
        \caption{#3}
        \label{#4}
    \end{figure}
}
