%!TEX root = ../thesis-main.tex

\chapter{Introduction}\label{chap:introduction}
The scientific method is characterised by the formulation of a \emph{theory}
which needs to be proven as false or true. This forms the first scientific
pillar, which is the theoretical assumption and knowledge used to describe a
certain phenomena, a property or a behaviour. The process of gathering evidence
to support the validity of the theory is what \emph{experiments} try to
accomplish, namely a series of observations of the real world that confirm or
refute the initial hypothesis. In the modern era of sciences, computer systems
backs the vast majority of modern scientific discoveries, whether through the
usage of the raw computing power (e.g. weather forecasting equations resolution
through Massively Parallel Computers) of such systems or by exploiting virtual
representations of the real world ~\cite{Post2005-kn}. With the widespread
adoption of simulation and simulators in the scientific method, researchers are
now considering simulation as the \emph{third pillar} of science which
complements theory and experiments. Some argues that simulation cannot be
compared to experimentation, while others claim that there is more in common
between simulation and experimentation ~\cite{WinsbergEric2010Sita}.
In the general scientific method framework, theory provides the model of the
hypothesis, while through experiments some may validate those models against
reality, while simulation can serve as sources of new insights or new
hypothesis which will be tested experimentally. Therefore, simulation is
essential in most scientific development workflows, and modern literature has
plenty of successful applications of simulations in various fields from biology
to physics and from chemistry to social sciences.

\section{Simulation}\label{sec:simulation}
A \emph{simulation} is the imitation of the operation of a real-world process
or system over time ~\cite{BanksDESS10}. A simulation is built upon a
\emph{model} described through a set of mathematical, logical and symbolical
relationships between entities in the system. The model is then simulated
inside the system over time, making it possibile for observers to make analysis
and prediction of model's impact on system performance. Moreover, the model is
defined as a representation of a system for the purpose of studying that system,
meaning that the set of characteristics of the model will include only those
details useful in expanding or exploring the knowledge of the original problem.
Simulation can also serve the purpose in giving insights on the model's
performance with respect to the system, actively driving the design process of
such model before it is even built in reality. A \emph{system} in simulation
terms, is defined as a set of objects which can interact with each other toward
the accomplishment of some purpose. An important separation must be drawn in
the boundary between the system and its environment, where the latter can
influence the system (changes in the \emph{system environment}) and viceversa.

%================================================================================================
\begin{figure}[h]
	\centering
	\begin{tikzpicture}[
			node distance=1.2em and 2em,
			box/.style={
					draw, thick, rounded corners,
					align=center,
					minimum width=0.32\linewidth,
					minimum height=3em
				},
			smallbox/.style={
					draw, thick, rounded corners,
					align=center,
					minimum width=0.32\linewidth,
					minimum height=2.5em
				},
			arrow/.style={->, thick}
		]

		% Left column
		\node[box] (attributes) {Attributes};
		\node[box, below=of attributes] (activities) {Activities};
		\node[
			draw, rounded corners,
			inner sep=1.2em,
			fit=(attributes)(activities),
			label={[anchor=north]north:\textbf{Entity}}
		] (entity) {};

		\node[
			draw, rounded corners,
			minimum width=0.32\linewidth,
			minimum height=3em,
			align=center,
			below=2em of entity
		] (entity2) {\textbf{Entity}};

		% Right column
		\node[box, right=4em of entity] (state) {State};
		\node[box, below=3em of state] (endo) {Endogenous Events};

		% System boundary
		\node[
			draw, thick, dashed, rounded corners,
			fit=(entity)(entity2)(attributes)(activities)(state)(endo),
			inner sep=1.2em,
			label={[anchor=north west]north west:\textbf{System}}
		] (system) {};

		% Environment
		\node[smallbox, above=5em of state] (exo) {Exogenous Events};

		\node[
			draw, thick, rounded corners,
			fit=(system)(exo),
			label={[anchor=north west]north west:\textbf{Environment}}
		] (environment) {};

		% Arrows
		\draw[arrow] (entity) -- (state);
		\draw[arrow] (attributes) -- (state);
		\draw[arrow] (activities) -- (state);
		\draw[arrow] (entity2) -- (state);

		\draw[arrow] (endo) -- node[midway, right] {influence} (state);
		\draw[arrow] (exo) -- node[midway, right] {influence} (state);

		\draw[arrow, <->] (entity) -- node[midway, left] {interaction} (entity2);

	\end{tikzpicture}
	\caption{System, environment, and state evolution in discrete-event modeling.}
	\label{fig:des-core}
\end{figure}
%================================================================================================

Inside the environment, \emph{entities} interact with each other and can be
described by a set of \emph{attributes} and their behaviour is represented by
\emph{activities} which last over time. The set of variables on a certain time
instant during a simulation is used to define the system \emph{state}, which in
turn can be influenced by internal (\emph{endogenous}) or external
(\emph{exogenous}) events. The interaction between the components is summarised
in \Cref{fig:des-core}.

Furthermore, a simulation could be characterised by the time span or the time
instant in which a system is investigated, talking respectively about dynamic
or static (Monte Carlo simulation) simulation models. In addition to that, some
models may be represented by a deterministic set of input variables, whereas
the vast majority of interesting problems may be expressed by means of
stochastic models.

\subsection{Discrete-Event System Simulation}\label{ssec:ddess}
Discrete-Event System Simulation is the modeling of systems in which the state
variable changes only at a discrete set of points in time. The model is
evaluated by means of \emph{numerical methods}, i.e. system behaviours are
approximated through computational procedures, rather than ``solving'' the
equations representing system behaviours. Most applications of \ac{DES} deals
exclusively with dynamic, stochastic systems that change in a discrete manner,
i.e. the state variable changes only at a discrete set of points in time. Those
points are represented by events occurring. This means that a \ac{DES} proceeds
by producing a sequence of system snapshots that represent the evolution of the
system through time including, for each $t$ so that $\text{CLOCK} = t$, the state
at time $t$ along with the list of all activities in progress - the \ac{FEL}.

\subsubsection{Event Scheduling}
Event scheduling and the advancement of time is a mechanism to advance
simulation time, while granting that the chronological order of events is
correct In the context of stochastic discrete simulations, scheduling an event
means inserting an event in the \ac{FEL} computing its duration by drawing or
calculating a sample value from a statistical distribution. In this way at any
given time the \ac{FEL} is strictly ordered by event time (i.e.
chronologically). After the state is updated at the $t$-th snapshot, the
$\text{CLOCK}$ advances at simulation time $\text{CLOCK} = t_1$, the next event
is removed from the \ac{FEL} and executed, effectively creating the snapshot
for time $t_1$. This procedure repeats until the simulation is over.

While the event scheduling algorithm provides the logical framework for
\ac{DES}, its computational \emph{efficiency} is governed by how the \ac{FEL}
is managed. In systems with a massive number of events such as biochemical
systems, the overhead of searching, updating and keeping the \ac{FEL} sorted
becomes easily the primary bottleneck.

\subsection{The Stochastic Simulation Algorithm}
While general \ac{DES} frameworks often rely on empirical distributions for
event timings, the \ac{SSA} ~\cite{GillespieESSCCR:1977} provides a physically
grounded method for simulating the time-evolution of systems of \emph{chemical
	reactions}. Chemical systems in the "direct method" follows the
\emph{Markovian} property, which states that the probability of a reaction
occurring depends only on the current state and not on the history of the
system prior to that state. Assuming that the probability of any reaction
occurring follows a Poisson process, the probability for any reaction to be the
next one is computed using markovian rates, i.e.
$$ P(\text{next} = \mu) = \frac{a_{\mu}}{\sum_j a_j} $$
Having the markovian rate, the probability that a
reaction occurs at time $\tau$ is its speed times the probability distribution,
namely a negative exponential function whose exponent is the sum of the speeds
of all reactions in the system.
Moreover, the \emph{next reaction time} could
be similarly computed as a derived from the aggregate propensity $\lambda =
	\sum_j a_j$, and the probability that a reaction $R$ occurs in time $\tau$ i.e.
the \ac{PDF} of the exponential distribution: $P(\tau) =
	\lambda e ^{-\lambda \tau}$. Finally, it is possible to sample $\tau$  from
the \ac{PDF} using the \emph{Inverse Transform Sampling} by picking a
a uniformly distributed number $\rho \in [0, 1]$, then:

$$ \tau = \frac{-\ln(\rho)}{\lambda} $$

The algorithm can be summarised in the following pseudocode:

\begin{algorithm}
	\caption{Stochastic Simulation Algorithm}\label{alg:ssa}
	\begin{algorithmic}[1]
		\Procedure{SSA}{}
		\State $T \gets 0$
		\BState \emph{loop}:
		\For{$r \in R$}
		$\textit{compute } a_r$
		\EndFor
		\State Select next reaction $\mu$
		\State $r \gets P(r=\mu) = \frac{a_r}{\sum_{j \in R}a_j} $
		\State Execute the reaction
		\State $T \gets T - \frac{\ln(1-r)}{\lambda}$
		\EndProcedure
	\end{algorithmic}
\end{algorithm}
