%!TEX root = ../thesis-main.tex
\chapter{An Introduction to Simulation}\label{ch:introduction}
The scientific method is characterised by the formulation of a \emph{theory}
which needs to be proven as false or true. This forms the first scientific
pillar, which is the theoretical assumption and knowledge used to describe a
certain phenomena, a property or a behaviour. The process of gathering evidence
to support the validity of the theory is what \emph{experiments} try to
accomplish, namely a series of observations of the real world that confirm or
refute the initial hypothesis. In the modern era of sciences, computer systems
backs the vast majority of modern scientific discoveries, whether through the
usage of the raw computing power (e.g. weather forecasting equations resolution
through Massively Parallel Computers) of such systems or by exploiting virtual
representations of the real world ~\cite{Post2005-kn}. With the widespread
adoption of simulation and simulators in the scientific method, researchers are
now considering simulation as the \emph{third pillar} of science which
complements theory and experiments. Some argues that simulation cannot be
compared to experimentation, while others claim that there is more in common
between simulation and experimentation ~\cite{WinsbergEric2010Sita}.
In the general scientific method framework, theory provides the model of the
hypothesis, while through experiments some may validate those models against
reality, while simulation can serve as sources of new insights or new
hypothesis which will be tested experimentally. Therefore, simulation is
essential in most scientific development workflows, and modern literature has
plenty of successful applications of simulations in various fields from biology
to physics and from chemistry to social sciences.

\section{Simulation and Modelling}
\subsection{System Modelling concepts}
\subsection{The Simulation Framework}

\section{Simulation Formalisms}
\subsection{Discrete Time Simulators}

The Discrete Time Model assumes a \emph{stepwise} mode of execution where, at a
particular time instant, the model is in a particular state and it defines how
this state changes. Based on its current state and the current input, the model
can determine which will be its state and output in the next time instant.
\emph{Time} in discrete time models advances in discrete steps of integers
multiple of some basic period such as 1 second, 1 day or 1 year.

A discrete time model determines its next state and output at a given time
$t$ with the following functions:

\begin{subequations}
	 \begin{align}
		q(t + 1) & = \delta(q(t), x(t)) \label{eq:state_transition} \\
		y(t) & = \lambda(q(t), x(t)) \label{eq:output_transition}
	 \end{align}
\end{subequations}
Starting from an initial state $q(0)$, we determine the \emph{state trajectory}
$q(0), q(1), \dots$ by applying the \emph{state transition function}
\eqref{eq:state_transition} ; similarly, by applying the \emph{output function}
\eqref{eq:output_transition} to the initial state, we obtain the \emph{output
trajectory}.

\ac{DES}

\subsection{Differential Equation Simulators}
\subsection{Discrete Event Simulators}
